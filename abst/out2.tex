\documentclass[a4j,twocolumn]{jsarticle}
\usepackage{subfigure}

\def\Vec#1{\mbox{\boldmath $#1$}}
\usepackage[dvipdfmx]{graphics}

\setlength{\textheight}{275mm}
\headheight 5mm
\topmargin -30mm
\textwidth 185mm
\oddsidemargin -15mm
\evensidemargin -15mm
\pagestyle{empty}

\begin{document}
¥documentclass[a4j,twocolumn]{jsarticle}
¥usepackage{subfigure}

¥def¥Vec#1{¥mbox{¥boldmath $#1$}}
¥usepackage[dvipdfmx]{graphicx}

¥setlength{¥textheight}{275mm}
¥headheight 5mm
¥topmargin -30mm
¥textwidth 185mm
¥oddsidemargin -15mm
¥evensidemargin -15mm
¥pagestyle{empty}
!ようこそ.
!!更新履歴
{{bbs}}
!参考文献テンプレ改 - Donkey (2016-08-30 (火) 14:41:35)
<<<tcsh
\bibliographystyle{jplain}
%\renewcommand{\bibname}{参考文献}
\begin{thebibliography}{1}
\bibitem{kasaharu}「今からでも始められる Chef 入門」,kasaharu,http://qiita.com/kasaharu/items55a3000db31c52ce0bd7,
2015/9/3アクセス.

\bibitem{yoshiba}「Chef実践入門」,吉羽 龍太郎, 安藤 祐介, 伊藤 直也, 菅井 祐太朗, 並河 祐貴,(技術評論社 2014).

\bibitem{schoolwith}「今更聞けない人の為の Chef 再入門」,School With ,http://blog.schoolwith.me/chef-re-introduction/,2015/9/10アクセス.

\end{thebibliography}

\end{document}
>>>
{{comment}}

!Donkey - 参考文献テンプレ (2016-08-30 (火) 14:40:26)
\bibliographystyle{jplain}
%\renewcommand{\bibname}{参考文献}
\begin{thebibliography}{1}
\bibitem{kasaharu}「今からでも始められる Chef 入門」,kasaharu,http://qiita.com/kasaharu/items55a3000db31c52ce0bd7,
2015/9/3アクセス.

\bibitem{yoshiba}「Chef実践入門」,吉羽 龍太郎, 安藤 祐介, 伊藤 直也, 菅井 祐太朗, 並河 祐貴,(技術評論社 2014).

\bibitem{schoolwith}「今更聞けない人の為の Chef 再入門」,School With ,http://blog.schoolwith.me/chef-re-introduction/,2015/9/10アクセス.
\end{thebibliography}

\end{document}
{{comment}}

!無題 - Donkey (2016-08-29 (月) 16:41:14)
hiki2latexのインストール
sudo gem install hiki2latex

実行方法
data/text/の後ろは自分が作成したページ名
hiki2latex Sites/hiki-1.0/data/text/GraduationThesis > out.tex

作成して出来上がったファイルの上の方を以下に書き換える(\begin{document}の上)

<<<tcsh
\documentclass[a4j,twocolumn]{jsarticle}
\usepackage{subfigure}

\def\Vec#1{\mbox{\boldmath $#1$}}
\usepackage[dvipdfmx]{graphicx}

\setlength{\textheight}{275mm}
\headheight 5mm
\topmargin -30mm
\textwidth 185mm
\oddsidemargin -15mm
\evensidemargin -15mm
\pagestyle{empty}
>>>

{{comment}}

!yamane_Memo - Donkey (2016-08-22 (月) 12:44:08)
string.each_char.select{|char| !(/[ -~。-゚]/.match(char))}.count
{{comment}}

!HikiStart - Donkey (2016-02-12 (金) 19:19:02)
HikiStartにHiki導入用のシステムを追加.
TestSettingのダウンロードにより, テスト機能の実装もおこなえる.
{{comment}}

!ShunkunTypeについて - Donkey  (2015-11-17 (火) 16:43:45)
[[ShunkunType]]にRakeコマンドのまとめを作成.

{{comment}}

!latexプラグイン実装(仮) - Donkey (2015-10-14 (水) 07:52:05)
とりあえず,localhostにlatex.rb実装.

pdf変換の実装とnishitani0用の調整しないと.
{{comment}}


!!近々にやらなければならないこと一覧
!!!レシピ関係
*rbenv.
*ruby(windowsも含めた).
*Ohaiってなんなのか.

!!!卒業研究関連
*現在の機能をRails版へ移行
*依存性も考慮したレシピの作成(とりあえず,手動).
*新機能の考案と実装(text読み込んでレシピ作成).

!!要リンクページ
{{orphan}}
