\documentclass[a4j,twocolumn]{jsarticle}
\usepackage{subfigure}

\def\Vec#1{\mbox{\boldmath $#1$}}
\usepackage[dvipdfmx]{graphics}

\setlength{\textheight}{275mm}
\headheight 5mm
\topmargin -30mm
\textwidth 185mm
\oddsidemargin -15mm
\evensidemargin -15mm
\pagestyle{empty}
\begin{document}

\title{小傾角粒界エネルギーの視覚化を容易にするソフト開発}
\author{関西学院大学 情報科学科 西谷研究室 1549 成田大樹}
\date{}
\maketitle

\section{はじめに}
西谷研究室では,Read-Shockleyと大槻による小傾角粒界エネルギーの実験結果の矛盾を解くために様々な手法をこれまで試してきたが,未だ解明されていない.
実験結果の具体的な矛盾点としては,小傾角粒界エネルギーの0度,及び90度に相当する立ち上がりの傾き方である.Hassonらによるシミュレーション結果では,0度,及び90度でそれぞれ異なる傾きになった.それに対して,大槻のシミュレーション結果では,小傾角粒界エネルギーの0度,及び90度に相当する立ち上がりの傾きが左右対称となった.そこで本研究では,原子の配置や粒界エネルギーの高低差を視覚的に検証し易くするためのソフトを開発する.

\section{現状}
実験結果の矛盾を解くために,第一原理計算ソフトVASPや原子間ポテンシャルを使ったシミュレーションをはじめ,Sutton Vitekによる粒子モデルの研究を取り組んできた.粒子シミュレーションのモデル作成アルゴリズムをおこなった八幡の研究では,最安定の原子配置での計算ができず,実験データはRead-Shockleyの理論予測となり,矛盾点を解明できなかった\cite{yahata}.

また,岩佐の研究では,最安定な原子配置をとるために原子の削除操作をおこない,第一原理計算ソフトVASPを用いて構造緩和し,系全体のエネルギーを計算した.この研究により,周期的境界条件の中で原子の削除操作ができ,予測通りに小傾角粒界エネルギーが低い結果となったが,"Viewer"による安定構造の検証をしなかったために,構造緩和に過ちを生じた\cite{iwasa}.

\section{手法}
本研究のソフト開発は,MVCモデルで作成していく.MVCモデルは,データ処理をおこなう"Model",処理結果を画面表示する"Viewer",リクエストを受け取って指示を出す"Controller"で構成されており,各々の機能が明確に分離している,機能が分かれていることにより,開発作業の分業化が容易におこなえるため, 視覚化するための機能に特化した開発が可能となる.

小傾角粒界の原子モデルを視覚的に確かめるためのモデル"Viewer"は,POSCAR形式のPOSCAR.txtで作成し.SVGで表示する.SVGを使用することで,ベクトルベースによる曲線や文字が描画しやすく,様々な画像フォーマットで描画することが可能になる.なお,粒界原子の構造は,Ruby言語で視覚化を容易に実現できる2次元画像描画ライブラリ"Cairo"を用いて作成する\cite{sudoh}.

\section{今後の課題}
岩佐の研究において,原子がz軸方向に動いていることを表示しなかったことが原因で構造緩和の過ちが生じたため,本研究では原子の配置と小傾角粒界エネルギーの大きさを多面的角度で描画できるように開発する.

\begin{thebibliography}{9}
\bibitem{yahata} 小傾角粒界粒子シミュレーションの原子ポテンシャル依存性, 八幡裕也 (関西学院大学 理工学部研究科情報科学専攻 修士論文 2015).
\bibitem{iwasa} 原子削除操作を加えた対称傾角粒界のエネルギー計算, 岩佐 恭佑(西学院大学 理工学部研究科情報科 学士論文 2016). 
\bibitem{sudoh} cairo:2次元画像描画ライブラリ,須藤功平, Rubyist Magazine - るびま, Vol.54 (2016-08).
\end{thebibliography}

\end{document}
