\documentclass[a4j,twocolumn]{jsarticle}
\usepackage{amssymb} % 高度な数式を表記するために使用
\usepackage[dvipdfmx]{graphicx}		% 図を入れるときに使用
\usepackage{wrapfig}		% 図の周りに本文を流し込みたいときに使用
\usepackage{here}
\usepackage{subfigure}

\def\Vec#1{\mbox{\boldmath $#1$}}
\usepackage[dvipdfmx]{graphics}

\setlength{\textheight}{275mm}
\headheight 5mm
\topmargin -30mm
\textwidth 185mm
\oddsidemargin -15mm
\evensidemargin -15mm
\pagestyle{empty}
\begin{document}

\title{小傾角粒界エネルギーの視覚化を容易にするソフト開発}
\author{関西学院大学 情報科学科 西谷研究室 1549 成田大樹}
\date{}
\maketitle

\section{はじめに}
西谷研究室では,小傾角粒界エネルギーについて,Read-Shockleyによる理論と大槻による実験結果の矛盾を解くために様々な手法をこれまで試してきた.
この「矛盾点」は,小傾角粒界エネルギーの0度,及び90度に相当する立ち上がりの傾きの違いである.
図\ref{fig:one}に示したように,Hassonらによるシミュレーション結果では,0度,及び90度でそれぞれ異なる傾きになった.
それに対して,大槻の実験結果では,これらの傾きが左右対称となった.
本研究では,原子の配置や粒界エネルギーの高低差を視覚的に検証し易くするためのソフトを開発する.

\begin{figure}[h]
\begin{center}
\begin{tabular}{cc}
   \includegraphics[width=40mm]{./Hasson.png} &
   \includegraphics[width=40mm]{./Otsuki.png} \\
   (a) & (b)
\end{tabular}
  \caption{(a) Hassonと(b)大槻の求めたAl(100)対称傾角粒界エネルギーの方位依存性.}
  \label{fig:one}
\end{center}
\end{figure}


\vskip 1.5em%

\section{現状}
実験結果の「矛盾を解くため」に,
第一原理計算ソフトVASPや原子間ポテンシャルを使ったシミュレーションをはじめ,
Sutton Vitekによる粒子モデルの研究を取り組んできた.

経験的原子間ポテンシャルによるシミュレーションをおこなった八幡の研究では,
Read-Shockleyの理論予測と同様の結果となった\cite{yahata}.

一方,岩佐は,最安定な原子配置を探索するために原子の削除操作を取り入れ,
第一原理計算ソフトVASPを用いて構造緩和し,
系全体のエネルギーを計算した.
この研究では,
予測通りに小傾角粒界エネルギーが大槻の結果を再現する程度の低いエネルギーとなった\cite{iwasa}.
ところが,構造緩和に過ちが生じていた.
図\ref{fig:two}に示したとおり,原子が全体的に傾いて,粒界がより低い角度に鳴った状態を計算していたことになる.

これは,安定構造の原子位置を視覚的に確認をしなかったためである.
本研究では原子の配置を容易に表示するソフトを開発する.
\begin{figure}[h]
\begin{center}
   \includegraphics[width=80mm]{./Iwasa.png} 
     \caption{岩佐の失敗.}
  \label{fig:two}
\end{center}
\end{figure}



\section{手法}
本研究のソフト開発は,MVCモデルで作成していく.
MVCモデルとは,web applicationの開発において取られている手法であり,
データ処理をおこなう"Model",
処理結果を画面表示する"Viewer",
リクエストを受け取って指示を出す"Controller"で構成されており,
各々の機能が明確に分離された「直交化」が意識されている.
機能を分けることにより
開発作業の分業化が容易におこなえるため, 
私が担当する視覚化のための機能構築に特化した開発が可能となる.

小傾角粒界の原子モデルを視覚的に確かめるためのモデル"Viewer"は,
VASPの入出力で採用されているPOSCAR形式のファイルを入力とし,
SVGで出力する.
SVGは,
ベクトルベースによるので,曲線や文字の拡大縮小しても画質のクオリティが高く,
汎用性が高いので画像表示,処理,変換が容易である.
SVGの生成には,
Ruby言語で視覚化を容易に実現できる2次元画像描画ライブラリ"Cairo"を用いる\cite{sudoh}.


\begin{thebibliography}{9}
\bibitem{yahata} 小傾角粒界粒子シミュレーションの原子ポテンシャル依存性, 八幡裕也 (関西学院大学 理工学部研究科情報科学専攻 修士論文 2015).
\bibitem{iwasa} 原子削除操作を加えた対称傾角粒界のエネルギー計算, 岩佐 恭佑(西学院大学 理工学部研究科情報科 学士論文 2016). 
\bibitem{sudoh} cairo:2次元画像描画ライブラリ,須藤功平, Rubyist Magazine - るびま, Vol.54 (2016-08).
\end{thebibliography}

\end{document}
