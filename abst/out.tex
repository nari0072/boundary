\documentclass[a4j,twocolumn]{jsarticle}
\usepackage{subfigure}

\def\Vec#1{\mbox{\boldmath $#1$}}
\usepackage[dvipdfmx]{graphics}

\setlength{\textheight}{275mm}
\headheight 5mm
\topmargin -30mm
\textwidth 185mm
\oddsidemargin -15mm
\evensidemargin -15mm
\pagestyle{empty}
\begin{document}

小傾角粒界エネルギーの視覚化を容易にするソフト開発

\section{はじめに}
未だ,Read-Shockleyと大槻の双方による小傾角粒界エネルギーの実験結果における矛盾が解明されていない.具体的に,Hassonらによるシミュレーション結果では,小傾角粒界エネルギーの0度,及び90度に相当する立ち上がりの傾きが異なる形で表示された.これに対し,大槻のシミュレーション結果では,小傾角粒界エネルギーの0度,及び90度に相当する立ち上がりの傾きが左右対称であった.西谷研究室では,小傾角粒界エネルギーによるこの矛盾を解くために様々な手法をこれまで試してきたが,未だ解明されていない.そこで本研究では,原子の配置や粒界エネルギーの高低差を視覚的に検証し易くするためのソフト開発である.

\section{現状}
実験結果の矛盾を解くために,第一原理計算ソフトVASPや原子間ポテンシャルを使ったシミュレーションをはじめ,Sutton Vitekによる粒子モデルの研究を取り組んできた.粒子シミュレーションのモデル作成アルゴリズムをおこなった八幡の研究では,再安定の原子配置での計算ができず,実験データはRead-Shockleyの理論予測となり,矛盾点を解明できなかった.また,岩佐は再安定な原子配置をとるために原子の削除操作をおこない,第一原理計算ソフトVASPを用いて構造緩和し,系全体のエネルギーを計算した.岩佐のこの研究は,周期的境界条件のなかで削除操作ができ,予測通りに小傾角粒界エネルギーが低い結果となったが,構造緩和に過ちがあったため,viewerによる安定構造の検証ができなかった.


\section{手法}
ソフトウェア設計は,機能場所が明確に分離でき,開発作業の分業化が比較的容易である,MVCモデルで作成する.

POSCAR形式のPOSCAR.txtを作成し,小傾角粒界の原子モデルを視覚的に確かめるためのモデルviewerをSVGで作成する.また,SVGの描写については,Rubyで2次元の視覚化を容易に実現する2次元画像描画ライブラリ”Cairo”を用いて,粒界原子の構造を表示する.Cairoを使用することで,ベクトルベースのため,曲線や文字を描画しやすく,様々な画像フォーマットで描画することが可能になる.



\section{今後の課題}
原子の配置と小傾角粒界エネルギーの高低差を多面的角度で描写することを簡易化する.これは,岩佐の研究において,原子がz軸方向に動いていることを表示できなかったことが構造緩和の過ちの原因であるためである,

参考文献
[1]小傾角粒界粒子シミュレーションの原子ポテンシャル依 存    性」, 八幡裕也 (関西学院大学 理工学部研究科情報科 学専攻 博士論文 2015).
[2]原子削除操作を加えた対称傾角粒界のエネルギー計算, 岩佐 恭佑(西学院大学 理工学部研究科情報科 学士論文 2016) 
[3]cairo:2次元画像描画ライブラリ,須藤功平
\end{document}
