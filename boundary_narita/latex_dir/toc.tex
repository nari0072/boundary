\section{1:序論}
\section{2:ソフト開発の手法}
\subsection{2.1:MVCモデルの概要と利点}
\subsection{2.2:SVG表示の特徴}
\subsection{2.3:rcairoを使用する利点}
\subsection{2.4:一般的な原子モデルソフトとの比較}
\section{3:ソフトの構成と描画}
\subsection{3.1:外部データの読み込み部}
\subsubsection{3.1.1:read\_pos}
\subsection{3.2:原子座標の計算部}
\subsubsection{3.2.1:identical\_atoms}
\subsubsection{3.2.2:mk\_deleted\_atom}
\subsection{3.3:粒界原子の描画部}
\subsubsection{3.3.1:draw\_backcolor, draw\_exes}
\subsubsection{3.3.2:open\_circle}
\subsubsection{3.3.3:draw\_each\_plane, pos\_y}
\subsubsection{3.3.4:draw\_atoms}
\subsubsection{3.3.5:find\_max}
\subsubsection{3.3.6:main\_draw}
\subsection{3.4:三面図による描画}
\section{4:結果と考察}
\subsection{4.1:各用途に合わせた原子配列の表示結果}
\subsubsection{4.1.1:構造緩和による原子移動の表示}
\subsubsection{4.1.2:削除された原子の識別表示}
\subsubsection{4.1.3:指定した層の白抜き表示}
\subsection{4.2:原子構造の改善点}
\subsection{4.3:考察と今後の課題}
\section{5:総括}
\section{6:参考文献}
