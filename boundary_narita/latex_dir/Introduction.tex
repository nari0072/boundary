
\section{概要}
本研究では,原子配列を容易に視覚化しやすくするためのソフト開発をおこなう.なお,ソフト開発はMVCモデルで作成していく.

\section{MVCモデル}
web applicationの開発において取られている手法であり,
データを処理するModel,画面に結果を出力するView,処理機能を制御するControlerの機能が明確に分離されている.

MVCモデルをおこなうことで,機能を直交化することができ,開発作業の分業化しやすく, 処理結果を画面表示する機能構築に特化した開発が可能となる.

