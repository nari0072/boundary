
\section{序論}
本研究の目的は,小傾角粒界エネルギーにおいてHassonらによるMorseポテンシャルを用いたシミュレーションの結果と大槻による実験結果の矛盾を原子レベルシミュレーションで解明することである.
両者の具体的な相違点として,Hassonらによるシミュレーション結果では,0度及び90度付近における対称傾角粒界エネルギーの立ち上がりがそれぞれ異なる傾きになった.
その一方,大槻の実験結果では,0度及び90度付近の対称傾角粒界エネルギーの傾きが左右対称になった.

\begin{figure}[htbp]\begin{center}
\includegraphics[width=6cm,bb=0 0 442 500]{../figs/./boundary_narita.002.jpg}
\caption{Hassonらによるシミュレーション結果.}
\label{default}\end{center}\end{figure}
Hassonらによるシミュレーションは,Read-Shockleyの理論予測をもとにおこなった.Read-Shockleyの式は以下のようになる.

\begin{figure}[htbp]\begin{center}
\includegraphics[width=6cm,bb=0 0 442 500]{../figs/./boundary_narita.003.jpg}
\caption{大槻による実験結果.}
\label{default}\end{center}\end{figure}
小傾角粒界エネルギーによる2つの結果の矛盾を明白にするために,西谷研究室において様々な検証が今までおこなわれてきた.
はじめに,第一原理計算ソフトVASPによる計算をおこない,その後,原子間ポテンシャルを使ったシミュレーションやSutton Vitekによる粒子モデルの研究を取り組んできた.また,八幡の研究では,経験的原子間ポテンシャルによるシミュレーションをおこない,Read-Shockleyの理論予測と同様の結果となった[1].
さらに,岩佐の研究では,最安定な原子配置を探索するために原子の削除操作を取り入れ,第一原理計算ソフトVASPを用いて構造緩和し,系全体のエネルギー計算をおこなった.
その結果,小傾角粒界エネルギーが予測通りに大槻の結果よりも低いエネルギーの値となり,粒界エネルギーの立ち上がり0度付近が最安定の構造になった.
しかしながら,粒界がより低い角度になった状態を計算していたため,図{図:004}のように原子が全体的に傾いた構造になり,全ての構造が最安定であることの検証までには至らなかった.
エネルギー計算による誤った構造緩和は,安定構造の原子配置を視覚的に確認しなかったことが原因である[2].

\begin{figure}[htbp]\begin{center}
\includegraphics[width=6cm,bb=0 0 442 500]{../figs/./boundary_narita.004.jpg}
\caption{岩佐による誤った構造緩和の結果}
\label{default}\end{center}\end{figure}
この失敗の原因を踏まえて,原子配置をより簡易的で視覚化できるプログラムを作成していく.
したがって,粒界原子配列を正確で,且つ容易に視覚化できるためのソフトを開発することが本研究の目標である.
本研究では小傾角粒界の原子モデルの配列を2次元で描画するソフトを開発する.

