\documentclass[12pt,a4paper]{jsarticle}
\usepackage[dvipdfmx]{graphicx}
\usepackage[dvipdfmx]{color}
\usepackage{listings}
% to use japanese correctly, install jlistings.
\lstset{
  basicstyle={\small\ttfamily},
  identifierstyle={\small},
  commentstyle={\small\itshape\color{red}},
  keywordstyle={\small\bfseries\color{cyan}},
  ndkeywordstyle={\small},
  stringstyle={\small\color{blue}},
  frame={tb},
  breaklines=true,
  numbers=left,
  numberstyle={\scriptsize},
  stepnumber=1,
  numbersep=1zw,
  xrightmargin=0zw,
  xleftmargin=3zw,
  lineskip=-0.5ex
}
\lstdefinestyle{customCsh}{
  language={csh},
  numbers=none,
}
\lstdefinestyle{customRuby}{
  language={ruby},
  numbers=left,
}
\lstdefinestyle{customTex}{
  language={tex},
  numbers=none,
}
\lstdefinestyle{customJava}{
  language={java},
  numbers=left,
}
\begin{document}
\title{卒業論文\\
\vspace{4cm} 三面図を利用した粒界原子配列の視覚化}
\author{ 関西学院大学 理工学部 情報科学科\\\\1549 成田 大樹}
\date{\vspace{3cm} 2017年  3月\\
\vspace{3cm} 指導教員  西谷 滋人 教授}
\maketitle
\tableofcontents

\tableofcontents
\include{introduction}
\include{drawing_atoms}
\include{2223_calc_memo}
\include{conclusion}
\include{Read-Shockley_inference}
\include{Otuki_experiment}
\include{calculation_Yahata}
\include{calculation_Murakami}
\include{calculation_Iwasa}
2223モデルの計算結果を記す.
\begin{lstlisting}[style=]

(Al)4  (Fm-3m)
   1.00000000000000
    11.3867605961746481    0.0100935929319685   -0.0017461789180479
     0.0041105062482931    6.3305944602789737   -0.0005256693932857
    -0.0010552304551614   -0.0005576642430577    7.7999169997776718


-0.112458518905E+03

dF-31*(-3.739501247) %=> 3.4660198

3.4660198/(6.3305944602789737*7.7999169997776718)*1.60218*10/2
\end{lstlisting}
Full relaxの最安定energyは,0.5623126780
\begin{quote}\begin{verbatim}
    xx=0.9899786872, yy=1.007043146
\end{verbatim}\end{quote}
であるが,fixの-3\_1において,
\begin{quote}\begin{verbatim}
 -3    1   11.15697    6.34918    8.08280 -112.35392 0.5573720133794792
\end{verbatim}\end{quote}
とそれより小さな値が出ている.

もう少しその方向で上がるまで計算を追加する必要がある.

Absolute interfacial energies of [001] tilt and twist grain boundaries in copper.
Acta Metallurgica, Vol. 7, May 1959, 319.
N.A.Gjostein and F.N.Rhines, 
Fig.4. Dependencies of grain boundary energy on misorientation for [001] tilt boundaries at 1065C.  Solid line represents the curve calculated from equation (1), using the large angle parameters.
\begin{quote}\begin{verbatim}
#Murakami-0
13.000  0.227  -1.483  646.970  2851,435
10.811  0.189  -1.668  560.606  2971.081
 9.324  0.613  -1.816  540.909  3323.874

#Murakami_90
 7.838  0.137  -1.989  356.061  2602.806
 9.189  0.160  -1.830  400.000  2494.103
10.811  0.189  -1.668  431.818  2288.535
12.703  0.222  -1.506  490.909  2214.202
22.703  0.396  -0.926  671.212  1693.944

#Otsuki-0
10.000  0.174  -1.740  280.000  1609.195
 7.300  0.127  -2.064  230.000  1811.024
 5.000  0.087  -2.439  190.000  2183.908

#Otsuki-90
10.000  0.174  -1.740  280.000  1609.195
 7.460  0.130  -2.040  230.000  1769.231
 6.000  0.105  -2.256  207.000  1971.429
\end{verbatim}\end{quote}
"Asymmetric tilt grain boundary structure and energy in copper and aluminium",
M. A. TSCHOPPy and D. L. MCDOWELL, Philosophical Magazine, Vol. 87, No. 25, 1 September 2007, 3871-3892
\end{document}
