\documentclass[12pt,a4paper]{jsarticle}
\usepackage[dvipdfmx]{graphicx}
\usepackage[dvipdfmx]{color}
\usepackage{listings}
% to use japanese correctly, install jlistings.
\lstset{
  basicstyle={\small\ttfamily},
  identifierstyle={\small},
  commentstyle={\small\itshape\color{red}},
  keywordstyle={\small\bfseries\color{cyan}},
  ndkeywordstyle={\small},
  stringstyle={\small\color{blue}},
  frame={tb},
  breaklines=true,
  numbers=left,
  numberstyle={\scriptsize},
  stepnumber=1,
  numbersep=1zw,
  xrightmargin=0zw,
  xleftmargin=3zw,
  lineskip=-0.5ex
}
\lstdefinestyle{customCsh}{
  language={csh},
  numbers=none,
}
\lstdefinestyle{customRuby}{
  language={ruby},
  numbers=left,
}
\lstdefinestyle{customTex}{
  language={tex},
  numbers=none,
}
\lstdefinestyle{customJava}{
  language={java},
  numbers=left,
}
\begin{document}
\title{卒業論文\\
\vspace{4cm} 三面図を利用した粒界原子配列の視覚化}
\author{ 関西学院大学 理工学部 情報科学科\\\\1549 成田 大樹}
\date{\vspace{3cm} 2017年  3月\\
\vspace{3cm} 指導教員  西谷 滋人 教授}
\maketitle
\tableofcontents

\tableofcontents
\include{introduction}

\section{原子配列を表示するソフト"view"}
本研究で開発する"viewer"は,
小傾角粒界の原子モデルを視覚的に確かめるためのソフトであり,
VASPの入出力で採用されているPOSCAR形式のファイルを入力とし,
SVGで出力する.SVGには,以下の特徴がある.
ベクトルベースによるため,曲線や文字の拡大縮小しても画質が劣化することなく表示できる,
汎用性が高く,画像表示や変換の処理が容易にできる.
なお,SVGの生成には,Ruby言語で視覚化を容易に実現できる2次元画像描画ライブラリ"Cairo"を用いる.

\section{三面図の使用}
三面図は,立体を正面図,平面図,側面図の三方向からみて投影した図を展開したもので,立体の形状を2次元上で適切に表示することが出来る.
実際に,三面図を使用したPOSCAR\_2223の原子配置は図のように表示される.

\section{各関数の機能}
\subsection{read\_pos}
\subsection{identical\_atoms}
\subsection{mk\_deleted\_atom}
\subsection{draw\_exes}
\subsection{draw\_atoms}
\subsection{pos\_y}
\subsection{open\_circle}
\subsection{draw\_each\_plane}
\subsection{find\_max}
\subsection{main\_draw}

\begin{quote}\begin{verbatim}

(Al)4  (Fm-3m)                          
   1.00000000000000     
    11.3867605961746481    0.0100935929319685   -0.0017461789180479
     0.0041105062482931    6.3305944602789737   -0.0005256693932857
    -0.0010552304551614   -0.0005576642430577    7.7999169997776718


-0.112458518905E+03

dF-31*(-3.739501247) %=> 3.4660198

3.4660198/(6.3305944602789737*7.7999169997776718)*1.60218*10/2
\end{verbatim}\end{quote}
Full relaxの最安定energyは,0.5623126780
\begin{quote}\begin{verbatim}
    xx=0.9899786872, yy=1.007043146
\end{verbatim}\end{quote}
であるが,fixの-3\_1において,
\begin{quote}\begin{verbatim}
 -3    1   11.15697    6.34918    8.08280 -112.35392 0.5573720133794792
\end{verbatim}\end{quote}
とそれより小さな値が出ている.

もう少しその方向で上がるまで計算を追加する必要がある.
\begin{quote}\begin{verbatim}
cat 2223_ind.txt 
  -6    1   10.81190    6.34918    8.08280 -112.30642 0.5647867426403657
  -6    2   10.81190    6.41205    8.08280 -112.3123 0.5583407509605174
  -6    3   10.81190    6.47491    8.08280 -112.29352 0.5557945934739222
  -6    4   10.81190    6.53777    8.08280 -112.25456 0.5563566217861508


  -5    1   10.92692    6.34918    8.08280 -112.36269 0.5560030202085747
  -5    2   10.92692    6.41205    8.08280 -112.34929 0.5526232371583745
  -5    3   10.92692    6.47491    8.08280 -112.31201 0.5529643568367403
  -5    4   10.92692    6.53777    8.08280 -112.25093 0.5569069177758106


  -4    0   11.04194    6.28632    8.08280 -112.38303 0.5583562346401736
  -4    1   11.04194    6.34918    8.08280 -112.38191 0.5530027866002744
  -4    2   11.04194    6.41205    8.08280 -112.35048 0.5524392998395155
  -4    3   11.04194    6.47491    8.08280 -112.29313 0.5558542901904441


  -3   -1   11.15697    6.22346    8.08280 -112.3816 0.5642239285248216
  -3    0   11.15697    6.28632    8.08280 -112.37952 0.5589096232023381
  -3    1   11.15697    6.34918    8.08280 -112.35392 0.5573720133794792
  -3    2   11.15697    6.41205    8.08280 -112.30236 0.5598771685650941


  -2   -2   11.27199    6.16059    8.08280 -112.38376 0.5696338188082548
  -2   -1   11.27199    6.22346    8.08280 -112.38698 0.563367146901053
  -2    0   11.27199    6.28632    8.08280 -112.36558 0.5611074113950909
  -2    1   11.27199    6.34918    8.08280 -112.32077 0.562546713905762
  -2    2   11.27199    6.41205    8.08280 -112.25037 0.5679132199998959


  -1   -2   11.38701    6.16059    8.08280 -112.37613 0.5708613188749168
  -1   -1   11.38701    6.22346    8.08280 -112.36337 0.5671271123540664
  -1    0   11.38701    6.28632    8.08280 -112.32574 0.5673886080664855
  -1    1   11.38701    6.34918    8.08280 -112.26303 0.5715599026957316
  -1    2   11.38701    6.41205    8.08280 -112.17362 0.5797764042203458


   0   -2   11.50203    6.16059    8.08280 -112.33666 0.5772111783285478
   0   -1   11.50203    6.22346    8.08280 -112.3085 0.5758653293979867
   0    0   11.50203    6.28632    8.08280 -112.25474 0.5785825077569731
   0    1   11.50203    6.34918    8.08280 -112.17344 0.5855448625795808
   0    2   11.50203    6.41205    8.08280 -112.07095 0.5956460214700341


   1   -2   11.61705    6.16059    8.08280 -112.26554 0.5886528303260675
   1   -1   11.61705    6.22346    8.08280 -112.22402 0.5893190304344416
   1    0   11.61705    6.28632    8.08280 -112.15653 0.5940663512584203
   1    1   11.61705    6.34918    8.08280 -112.06291 0.6027985473207549
   1    2   11.61705    6.41205    8.08280 -111.94264 0.6154787923798172


   2   -2   11.73207    6.16059    8.08280 -112.16641 0.604600678505402
   2   -1   11.73207    6.22346    8.08280 -112.10922 0.6076012851572761
   2    0   11.73207    6.28632    8.08280 -112.02921 0.6141396936329679
   2    1   11.73207    6.34918    8.08280 -111.92426 0.6244417517843739
   2    2   11.73207    6.41205    8.08280 -111.79417 0.6384276778678974
\end{verbatim}\end{quote}
だが,最安定は,
\begin{quote}\begin{verbatim}
 -4    2   11.04194    6.41205    8.08280 -112.35048 0.5524392998395155
\end{verbatim}\end{quote}
その周りも同じ程度の値.


\section{総括}
 本研究は,最安定の構造をとるためにおこなう原子の削除操作,及び構造緩和を視覚的に,且つ容易に把握できるソフトを開発することが目的であった.
作成したソフトは,rcairoを用いて2次元で小傾角粒界の原子配列を描画し,SVG形式の三面図で表示するようにした.\\
 開発を進めた結果,様々な用途に合わせて原子配列を表示することが可能となった.
まず,削除操作を表した原子配列の表示では,削除の有無により原子の色と大きさを変えて描画したことで,削除された原子の個数,並びに各位置を視覚的に把握することができた.
また,構造緩和による原子移動の表示では,構造緩和前後で原子が移動した経路を示すことができ,構造緩和に過ちが生じていないかを容易に確認できるようになった.
さらに,指定したz軸の層の原子を白抜きする機能を追加したことで,上面から見た各層の原子位置を正確に把握できるようになった.
三面図による描画,並びにVESTAによる視覚的検証の結果として,構造緩和前後の原子座標を格納したPOSCARファイル内に原子が一つ不足していることが発見できた.

 今後は,粒界原子配列の構造緩和をおこなう計算を見直す研究を進めていく必要がある.
さらに,本研究では,一つの原子配列ファイルをもとに様々な表示ができるソフトを開発したが,このソフトをより大きな粒界原子配列を表示できる機能へ改良していかなければならない.



\section{Read-Shockleyの理論}
小傾角粒界の粒界エネルギーは,Read-Shockleyによって提案された転位が等間隔に並んだモデルによって計算できる.
このモデルの特徴のひとつとして,(001)tilt粒界において,幾何学的に要求される転位のバーガースベクトルが違うため,その立ち上がり角度に違いが出ることがある.図に示す通り,
eta=0近傍では,fccのユニットセルの大きさ$a$に等しいバーガースベクトルのずれによ
って粒界の両側にある原子の並びが一致する.一方で,$\theta=90^\circ$近傍では(110)方向粒界があるため,幾何学的に要求されるバーガースベクトルは$a/\sqrt(2)$と,小さくなる.
Read-Shockleyが転位論から導いた粒界エネルギーの理論式の導出は次の通りである.
このようにして求められたエネルギーは傾角が小さい領域では,$E_0$に比例するが,
これは,$E_0=\frac{b \tau_0}{2}$となり,バーガースベクトル$b$に比例している.この理論
予測は,最近行われたEAM経験的ポテンシャルを用いたTschopp-Mcdowellによるシミュレーション結果によっても再現されており,信頼できる結果として広く認知されている.

対称傾角粒界のエネルギーの角度 0度,及び 90度 に相当する立ち上がりが異なる結果となった.


a


test


test



\section{計算方法}
原子モデルの削除操作を行い最安定の構造を求める研究をおこなった.
小傾角粒界の原子モデルの作成には,西谷研究室で開発された3つのツールを使用した.
3つのツールの各機能は,具体的に以下の通りである.

拡張するサイズや傾ける角度を指定して対称傾角粒界を作成するmaker,

小傾角粒界の原子モデルを描画出力するviewer

エネルギーの高い原子で原子の配置番号が奇数のものを削除するadjuster

また,小傾角粒界の構造緩和は,第一原理計算ソフトVASPを使用し,系全体のエネルギーを計算した.

\section{研究結果と反省}
削除を用いた本シ ミュレーションでは大槻の実験結果を支持する値となり,大槻の原子モデルよりも安定した構造が得られた. 
したがって,粒界エネルギーの立ち上がり0度付近の最安定の構造を求める目的は達成できた. 
しかし 90度付近の傾きの結果は得られていないので矛盾を解明することはできなかった.
これは,粒界がより低い角度になった状態を計算しており,原子が全体的に傾いた構造になっていたためである.
この構造緩和の過ちは,安定構造の原子配置を視覚的に確認しなかったことで生じた.


2223モデルの計算結果を記す.
\begin{lstlisting}[style=]

(Al)4  (Fm-3m)
   1.00000000000000
    11.3867605961746481    0.0100935929319685   -0.0017461789180479
     0.0041105062482931    6.3305944602789737   -0.0005256693932857
    -0.0010552304551614   -0.0005576642430577    7.7999169997776718


-0.112458518905E+03

dF-31*(-3.739501247) %=> 3.4660198

3.4660198/(6.3305944602789737*7.7999169997776718)*1.60218*10/2
\end{lstlisting}
Full relaxの最安定energyは,0.5623126780
\begin{quote}\begin{verbatim}
    xx=0.9899786872, yy=1.007043146
\end{verbatim}\end{quote}
であるが,fixの-3\_1において,
\begin{quote}\begin{verbatim}
 -3    1   11.15697    6.34918    8.08280 -112.35392 0.5573720133794792
\end{verbatim}\end{quote}
とそれより小さな値が出ている.

もう少しその方向で上がるまで計算を追加する必要がある.

Absolute interfacial energies of [001] tilt and twist grain boundaries in copper.
Acta Metallurgica, Vol. 7, May 1959, 319.
N.A.Gjostein and F.N.Rhines, 
Fig.4. Dependencies of grain boundary energy on misorientation for [001] tilt boundaries at 1065C.  Solid line represents the curve calculated from equation (1), using the large angle parameters.
\begin{quote}\begin{verbatim}
#Murakami-0
13.000  0.227  -1.483  646.970  2851,435
10.811  0.189  -1.668  560.606  2971.081
 9.324  0.613  -1.816  540.909  3323.874

#Murakami_90
 7.838  0.137  -1.989  356.061  2602.806
 9.189  0.160  -1.830  400.000  2494.103
10.811  0.189  -1.668  431.818  2288.535
12.703  0.222  -1.506  490.909  2214.202
22.703  0.396  -0.926  671.212  1693.944

#Otsuki-0
10.000  0.174  -1.740  280.000  1609.195
 7.300  0.127  -2.064  230.000  1811.024
 5.000  0.087  -2.439  190.000  2183.908

#Otsuki-90
10.000  0.174  -1.740  280.000  1609.195
 7.460  0.130  -2.040  230.000  1769.231
 6.000  0.105  -2.256  207.000  1971.429
\end{verbatim}\end{quote}
"Asymmetric tilt grain boundary structure and energy in copper and aluminium",
M. A. TSCHOPPy and D. L. MCDOWELL, Philosophical Magazine, Vol. 87, No. 25, 1 September 2007, 3871-3892
\end{document}
