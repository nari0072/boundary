
\section{原子配列を表示するソフト"view"}
本研究で開発する"viewer"は,
小傾角粒界の原子モデルを視覚的に確かめるためのソフトであり,
VASPの入出力で採用されているPOSCAR形式のファイルを入力とし,
SVGで出力する.SVGには,以下の特徴がある.
ベクトルベースによるため,曲線や文字の拡大縮小しても画質が劣化することなく表示できる,
汎用性が高く,画像表示や変換の処理が容易にできる.
なお,SVGの生成には,Ruby言語で視覚化を容易に実現できる2次元画像描画ライブラリ"Cairo"を用いる.

\section{三面図の使用}
三面図は,立体を正面図,平面図,側面図の三方向からみて投影した図を展開したもので,立体の形状を2次元上で適切に表示することが出来る.
実際に,三面図を使用したPOSCAR\_2223の原子配置は図のように表示される.

\section{各関数の機能}
\subsection{read\_pos}
\subsection{identical\_atoms}
\subsection{mk\_deleted\_atom}
\subsection{draw\_exes}
\subsection{draw\_atoms}
\subsection{pos\_y}
\subsection{open\_circle}
\subsection{draw\_each\_plane}
\subsection{find\_max}
\subsection{main\_draw}
